%%%%%%%%%%%%%%%%%%%%%%%%%%%%%%%%%%%%%%%%% 
% Beamer Presentation
% LaTeX Template
% Version 1.0 (10/11/12)
% 
% This template has been downloaded from:
% http://www.LaTeXTemplates.com
% 
% License:
% CC BY-NC-SA 3.0 (http://creativecommons.org/licenses/by-nc-sa/3.0/)
% 
%%%%%%%%%%%%%%%%%%%%%%%%%%%%%%%%%%%%%%%%% 

% ----------------------------------------------------------------------------------------
%	PACKAGES AND THEMES
% ----------------------------------------------------------------------------------------

\documentclass[cjk]{beamer}

\mode<presentation> {

  % The Beamer class comes with a number of default slide themes
  % which change the colors and layouts of slides. Below this is a list
  % of all the themes, uncomment each in turn to see what they look like.

  % \usetheme{default}
  % \usetheme{AnnArbor}
  % \usetheme{Antibes}
  % \usetheme{Bergen}
  % \usetheme{Berkeley}
  % \usetheme{Berlin}
  % \usetheme{Boadilla}
  % \usetheme{CambridgeUS}
  % \usetheme{Copenhagen}
  % \usetheme{Darmstadt}
  % \usetheme{Dresden}
  % \usetheme{Frankfurt}
  % \usetheme{Goettingen}
  % \usetheme{Hannover}
  % \usetheme{Ilmenau}
  \usetheme{JuanLesPins}
  % \usetheme{Luebeck}
  % \usetheme{Madrid}
  % \usetheme{Malmoe}
  % \usetheme{Marburg}
  % \usetheme{Montpellier}
  % \usetheme{PaloAlto}
  % \usetheme{Pittsburgh}
  % \usetheme{Rochester}
  % \usetheme{Singapore}
  % \usetheme{Szeged}
  % \usetheme{Warsaw}

  % As well as themes, the Beamer class has a number of color themes
  % for any slide theme. Uncomment each of these in turn to see how it
  % changes the colors of your current slide theme.

  % \usecolortheme{albatross}
  % \usecolortheme{beaver}
  % \usecolortheme{beetle}
  % \usecolortheme{crane}
  % \usecolortheme{dolphin}
  % \usecolortheme{dove}
  % \usecolortheme{fly}
  % \usecolortheme{lily}
  % \usecolortheme{orchid}
  % \usecolortheme{rose}
  % \usecolortheme{seagull}
  % \usecolortheme{seahorse}
  % \usecolortheme{whale}
  % \usecolortheme{wolverine}

  % \setbeamertemplate{footline} % To remove the footer line in all slides uncomment this line
  % \setbeamertemplate{footline}[page number] % To replace the footer line in all slides with a simple slide count uncomment this line

  % \setbeamertemplate{navigation symbols}{} % To remove the navigation symbols from the bottom of all slides uncomment this line
}

\usepackage{graphicx} % Allows including images
\usepackage{multirow}
\usepackage{booktabs} % Allows the use of \toprule, \midrule and \bottomrule in tables
\graphicspath{ {images/} }

\usepackage[encapsulated]{CJK}
\usepackage{ucs}
\usepackage[utf8x]{inputenc}

% ----------------------------------------------------------------------------------------
%	TITLE PAGE
% ----------------------------------------------------------------------------------------
\title[中国科学院通用芯片与基础软件研究中心 CPU and OS Research Center CAS]{验证平台组月度报告} % The short title appears at the bottom of every slide, the full title is only on the title page
\author{衣冠宇} % Your name
\institute[CAS] % Your institution as it will appear on the bottom of every slide, may be shorthand to save space
{
  中国科学院通用芯片与基础软件研究中心 \\ % Your institution for the title page
  \medskip
  验证平台组
  % \textit{yigy_cpu@sari.ac.cn} % Your email address
}
\date{2016年7月21日} % Date, can be changed to a custom date

\begin{document}
\begin{CJK}{UTF8}{gbsn}

  \begin{frame}
    \titlepage % Print the title page as the first slide
  \end{frame}

  % ----------------------------------------------------------------------------------------
  %	PRESENTATION SLIDES
  % ----------------------------------------------------------------------------------------

  % ------------------------------------------------
  \section{本月任务} % Sections can be created in order to organize your presentation into discrete blocks, all sections and subsections are automatically printed in the table of contents as an overview of the talk
  % ------------------------------------------------

  \begin{frame}
    \begin{enumerate}
    \item 验证平台组;
    \item 验证平台组。
    \end{enumerate}
  \end{frame}

  % ------------------------------------------------
  \section{本月任务}
  % ------------------------------------------------

  \begin{frame}
    \begin{block}{任务一}
      \begin{itemize}
      \item 验证平台组;
      \item 验证平台组;
      \item 验证平台组;
      \item 验证平台组;
      \item 验证平台组。
      \end{itemize}
    \end{block}
  \end{frame}

  % ------------------------------------------------
  \section{本月任务}
  % ------------------------------------------------

  \begin{frame}
    \begin{block}{任务二}
      \begin{itemize}
      \item 验证平台组;
      \item 验证平台组;
      \item 验证平台组。
      \end{itemize}
    \end{block}
  \end{frame}

  % ------------------------------------------------
  \section{下月计划} % Sections can be created in order to organize your presentation into discrete blocks, all sections and subsections are automatically printed in the table of contents as an overview of the talk
  % ------------------------------------------------

  \begin{frame}
    \begin{block}{任务一}
      \begin{itemize}
      \item 验证平台组;
      \item 验证平台组;
      \item 验证平台组。
      \end{itemize}
    \end{block}
    \begin{block}{任务二}
      \begin{itemize}
      \item 验证平台组;
      \item 验证平台组;
      \item 验证平台组。
      \end{itemize}
    \end{block}
  \end{frame}

  % ------------------------------------------------
  \section{Q \& A}
  % ------------------------------------------------

  \begin{frame}
    \Huge{\centerline{Q \& A}}
  \end{frame}

  % ----------------------------------------------------------------------------------------

\end{CJK}
\end{document} 
%%% Local Variables:
%%% mode: latex
%%% TeX-master: t
%%% End:
